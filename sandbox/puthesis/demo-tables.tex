%
%  demo-tables.tex  2013-03-29  Mark Senn  http://engineering.purdue.edu/~mark
%
%  Demonstrate how to do tables.
%

\chapter{Demonstrate Tables}

% \newlength{\ta}
% \newlength{\tb}
% \newlength{\tc}
% 
% \settowidth{\ta}{\vbox{\hbox{Money}\hbox{Market}}}
% \settowidth{\tb}{\vbox{\hbox{Stocks}\hbox{and}\hbox{Bonds}}}
% \settowidth{\tc}{\vbox{\hbox{Money}\hbox{Market}\hbox{and}\hbox{Stocks}}}
% 
% {
%     \renewcommand{\baselinestretch}{1}
%     \begin{table}
%       \caption{%
%         \hfil Allocation of the IRA and Keogh Wealth\hfil\break
%         \mbox{}\hfil for Investors With or Without Brokerage Accounts\hfil
%       }
%       \label{tab:ira}
%       \begin{center}
%         \begin{tabular}%
%           {%
%             |%
%             c%
%             |%
%             >{\centering\hspace{0pt}}m{\the\ta}%  Money Market
%             |%
%             c%                                    Stocks 
%             |%
%             c%                                    Bonds
%             |%
%             c%                                    Diversified
%             |%
%             >{\centering\hspace{0pt}}m{\the\tb}%  Stocks and Bonds
%             |%
%             >{\centering\hspace{0pt}}m{\the\tc}%  Money Market and Stocks
%             |%
%             c%                                    Others
%             |%
%           }
%           \hline
%           IMP&
%             Money Market&
%             Stocks&
%             Bonds&
%             Diversified&
%             Stocks and Bonds&
%             Money Market and Stocks&
%             Others\tabularnewline
%           \hline
%           1& 14.19\%& 57.71\%& 12.21\%& 4.50\%& 7.36\%& 3.04\%& 0.99\%\tabularnewline \hline
%           2& 14.08\%& 58.18\%& 12.32\%& 4.44\%& 7.30\%& 2.80\%& 0.88\%\tabularnewline \hline
%           3 &14.26\%& 58.09\%& 12.27\%& 4.50\%& 7.19\%& 2.75\%& 0.94\%\tabularnewline \hline
%           4 &13.94\%& 58.11\%& 12.14\%& 4.78\%& 7.35\%& 2.68\%& 0.99\%\tabularnewline \hline
%           5 &13.92\%& 58.13\%& 11.93\%& 4.56\%& 7.60\%& 2.98\%& 0.88\%\tabularnewline \hline
%         \end{tabular}
%       \end{center}
%       This table presents the allocations of the wealth in the IRA
%       and Keogh accounts in various asset classes.
%       Results from each set of imputed data are presented here.
%       The first column lists the number of the imputations,
%       and rest of the columns lists various allocations.
%       Entrees under each asset class show the percentage of investors
%       who have most of their IRA
%       and Keogh wealth invested in that particular asset class.
%       The asset class Diversified
%       includes stocks,
%       bonds,
%       and money market investments.
%       The asset class Others
%       include investments in various life insurance products,
%       annuities,
%       real estate, etc.
%       \medskip
%     \footnotesize SOURCE: Survey of Consumer Finances,
%     2001,
%     Federal Reserve Board,
%     USA.\par
%   \end{table}
% }

Here is a really simple table.

% "h" means put table here---don't let it float to top or bottom of page
\begin{table}[h]
  \caption{American Presidents}
  \begin{center}
    \begin{tabular}{rl}
      \bf Number& \bf Name\\
      1& George Washington\\
      2& John Adams\\
      3& Thomas Jefferson\\
    \end{tabular}
  \end{center}
  \label{ta:American-Presidents}
\end{table}

There are 72.27 points per inch.
I like to put 2 points of vertical space between the heading
(Number Name)
and the first line
(1 George Washington)
of the table.

\begin{table}[h]
  \caption{American Presidents with 2pt vertical space after heading}
  \begin{center}
    \begin{tabular}{rl}
      \bf Number& \bf Name\\[2pt]  % put 2pt vertical space after this line
      1& George Washington\\
      2& John Adams\\
      3& Thomas Jefferson\\
    \end{tabular}
  \end{center}
  \label{ta:American-Presidents-with}
\end{table}

\LaTeX\ can print horizontal and vertical rules in tables.
I don't like the way this looks.

\begin{table}[h]
  \caption{American Presidents with horizontal and vertical lines}
  \begin{center}
    % "|" prints a vertical rule, "c" means center
    \begin{tabular}{|c|l|}
      % "\hline" prints a horizontal rule
      \hline
      \bf\#& \bf Name\\
      \hline
      1& George Washington\\
      \hline
      2& John Adams\\
      \hline
      3& Thomas Jefferson\\
      \hline
    \end{tabular}
  \end{center}
  \label{ta:American-Presidents-with}
\end{table}

\newpage

Here is a more complicated table.

\begin{table}[h]
  \caption{C Bitwise Operators}
  \begin{center}
    % "|" prints a vertical rule, "c" means center
    \begin{tabular}{cccc}
      \bf A& \bf B& \bf A$|$B& \bf A\&B\\[2pt]
      0& 0& 0& 0\\
      0& 1& 1& 0\\
      1& 0& 1& 0\\
      1& 1& 1& 1\\
    \end{tabular}
  \end{center}
  \label{ta:C-Bitwise}
\end{table}

You can use Plain \TeX's \verb+\halign+ command to make tables also.
If you can't do a complicated table using \LaTeX\ commands
you may want to try using Plain \TeX\ commands.
\LaTeX's table making commands use Plain \TeX\ commands.

\begin{table}[h]
  \caption{American Presidents using {\tt\char'134 halign}}
  \hbox to \textwidth{\hss\vbox{\halign{%
    \strut #&      % 0. \strut
    \hfil#\qquad&  % 1. Number
    #\hfil\cr      % 2. Name
    %
    & \bf Number& \bf Name\cr
    \noalign{\vskip 2pt}
    & 1& George Washington\cr
    & 2& John Adams\cr
    & 3& Thomas Jefferson\cr
  }}\hss}
  \label{ta:American-Presidents-using}
\end{table}

The next page shows how to do a table that is too long to fit on one page.

\newpage

% This is loosely based on page 106 of _A Guide to LaTeX_, third edition,
% by Helmut Kopka and Patrick W. Daly.
\begin{longtable}{|l|l|}
    \caption{State Abbreviations}\\
    \hline
    State& Abbreviation\\
    \hline
  \endfirsthead
    \caption[]{\emph{continued}}\\
    \hline
    State& Abbreviation\\
    \hline
  \endhead
    \hline
    \multicolumn{2}{r}{\emph{continued on next page}}
  \endfoot
    \hline
  \endlastfoot
  Alabama& AL\\
  Alaska& AK\\
% American Samoa& AS\\
  Arizona& AZ\\
  Arkansas& AR\\
% Armed Forces Europe& AE\\
% Armed Forces Pacific& AP\\
% Armed Forces the Americas& AA\\
  California& CA\\
  Colorado& CO\\
  Connecticut& CT\\
  Delaware& DE\\
% District of Columbia& DC\\
% Federated States of Micronesia& FM\\
  Florida& FL\\
  Georgia& GA\\
% Guam& GU\\
  Hawaii& HI\\
  Idaho& ID\\
  Illinois& IL\\
  Indiana& IN\\
  Iowa& IA\\
  Kansas& KS\\
  Kentucky& KY\\
  Louisiana& LA\\
  Maine& ME\\
% Marshall Islands& MH\\
  Maryland& MD\\
  Massachusetts& MA\\
  Michigan& MI\\
  Minnesota& MN\\
  Mississippi& MS\\
  Missouri& MO\\
  Montana& MT\\
  Nebraska& NE\\
  Nevada& NV\\
  New Hampshire& NH\\
  New Jersey& NJ\\
  New Mexico& NM\\
  New York& NY\\
  North Carolina& NC\\
  North Dakota& ND\\
% Northern Mariana Islands& MP\\
  Ohio& OH\\
  Oklahoma& OK\\
  Oregon& OR\\
  Pennsylvania& PA\\
% Puerto Rico& PR\\
  Rhode Island& RI\\
  South Carolina& SC\\
  South Dakota& SD\\
  Tennessee& TN\\
  Texas& TX\\
  Utah& UT\\
  Vermont& VT\\
% Virgin Islands& VI\\
  Virginia& VA\\
  Washington& WA\\
  West Virginia& WV\\
  Wisconsin& WI\\
  Wyoming& WY\\
\end{longtable}

\newcommand{\cbackslash}{\char'134}
\newcommand{\copencurly}{\char'173}
\newcommand{\cclosecurly}{\char'175}

\newlength{\twidth}
\newlength{\theight}

\setlength{\twidth}{\textwidth}
\setlength{\theight}{\textheight}

\begin{sidewaystable}
  % The following two lines compensate for what I think is a bug.
  \setlength{\textwidth}{\theight}
  \setlength{\textheight}{\twidth}
  \caption{%
    sidewaystable
    {\tt\cbackslash begin\copencurly tabular\cclosecurly\/}%
    \ldots
    {\tt\cbackslash end\copencurly tabular\cclosecurly\/}%
  }
  \begin{center}
    \begin{tabular}{rl}
      \bf Number& \bf Name\\[2pt]  % put 2pt vertical space after this line
      1& George Washington\\
      2& John Adams\\
      3& Thomas Jefferson\\
    \end{tabular}
  \end{center}
\end{sidewaystable}

\begin{sidewaystable}
  % The following two lines compensate for what I think is a bug.
  \setlength{\textwidth}{\theight}
  \setlength{\textheight}{\twidth}
  \caption{%
    sidewaystable
    {\tt\cbackslash halign\copencurly}\ldots{\tt\cclosecurly\/} table%
  }
  \hbox to \textwidth{\hss\vbox{\halign{%
    \strut #&      % 0. \strut
    \hfil#\qquad&  % 1. Number
    #\hfil\cr      % 2. Name
    %
    & \bf Number& \bf Name\cr
    \noalign{\vskip 2pt}
    & 1& George Washington\cr
    & 2& John Adams\cr
    & 3& Thomas Jefferson\cr
  }}\hss}
\end{sidewaystable}

%\newlength{\ta}
%\settowidth{\ta}{\vbox{\hbox{Money}\hbox{Market}}}
%\newlength{\tb}
%\settowidth{\tb}{\vbox{\hbox{Stocks}\hbox{and}\hbox{Bonds}}}
%\newlength{\tc}
%\settowidth{\tc}{\vbox{\hbox{Money}\hbox{Market}\hbox{and}\hbox{Stocks}}}
%
%  {\renewcommand{\baselinestretch}{1}
%\begin{table}
%  \caption{\hfil Allocation of the IRA and Keogh Wealth\hfil\break\mbox{}\hfil for Investors With or Without Brokerage Accounts\hfil}
%  \label{tab:ira}
%  \begin{center}
%    \begin{tabular}%
%      {%
%        |%
%        c%
%        |%
%        >{\centering\hspace{0pt}}m{\the\ta}%  Money Market
%        |%
%        c%                                    Stocks 
%        |%
%        c%                                    Bonds
%        |%
%        c%                                    Diversified
%        |%
%        >{\centering\hspace{0pt}}m{\the\tb}%  Stocks and Bonds
%        |%
%        >{\centering\hspace{0pt}}m{\the\tc}%  Money Market and Stocks
%        |%
%        c%                                    Others
%        |%
%      }
%      \hline
%      IMP&
%        Money Market&
%        Stocks&
%        Bonds&
%        Diversified&
%        Stocks and Bonds&
%        Money Market and Stocks&
%        Others\tabularnewline
%      \hline
%      1& 14.19\%& 57.71\%& 12.21\%& 4.50\%& 7.36\%& 3.04\%& 0.99\%\tabularnewline \hline
%      2& 14.08\%& 58.18\%& 12.32\%& 4.44\%& 7.30\%& 2.80\%& 0.88\%\tabularnewline \hline
%      3 &14.26\%& 58.09\%& 12.27\%& 4.50\%& 7.19\%& 2.75\%& 0.94\%\tabularnewline \hline
%      4 &13.94\%& 58.11\%& 12.14\%& 4.78\%& 7.35\%& 2.68\%& 0.99\%\tabularnewline \hline
%      5 &13.92\%& 58.13\%& 11.93\%& 4.56\%& 7.60\%& 2.98\%& 0.88\%\tabularnewline \hline
%    \end{tabular}
%  \end{center}
%  This table presents the allocations of the wealth in the IRA
%  and Keogh accounts in various asset classes.
%  Results from each set of imputed data are presented here.
%  The first column lists the number of the imputations,
%  and rest of the columns lists various allocations.
%  Entrees under each asset class show the percentage of investors
%  who have most of their IRA
%  and Keogh wealth invested in that particular asset class.
%  The asset class Diversified
%  includes stocks,
%  bonds,
%  and money market investments.
%  The asset class Others
%  include investments in various life insurance products,
%  annuities,
%  real estate, etc.
%  \medskip
%  \footnotesize SOURCE: Survey of Consumer Finances,
%  2001,
%  Federal Reserve Board,
%  USA.\par
%\end{table}
%  }

