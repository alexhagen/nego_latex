%% LyX 2.0.6 created this file.  For more info, see http://www.lyx.org/.
%% Do not edit unless you really know what you are doing.
\documentclass[english,5p]{elsarticle}
\usepackage[T1]{fontenc}
\usepackage[latin9]{inputenc}
\usepackage{geometry}
\geometry{verbose,tmargin=2cm,bmargin=2cm,lmargin=2cm,rmargin=2cm,headsep=1cm}
\usepackage{array}
\usepackage{nomencl}
% the following is useful when we have the old nomencl.sty package
\providecommand{\printnomenclature}{\printglossary}
\providecommand{\makenomenclature}{\makeglossary}
\makenomenclature

\makeatletter

%%%%%%%%%%%%%%%%%%%%%%%%%%%%%% LyX specific LaTeX commands.
%% Because html converters don't know tabularnewline
\providecommand{\tabularnewline}{\\}

%%%%%%%%%%%%%%%%%%%%%%%%%%%%%% User specified LaTeX commands.
\usepackage{graphicx}
\usepackage{color}
\usepackage{minibox}
\usepackage{lipsum}

\makeatother

\usepackage{babel}
\begin{document}

\title{Design and Optimization of a Neutron Sensor Utilizing Acoustic Fields
- Experimentation and Simulation Using COMSOL and MCNP Physics Platforms}


\author[mfarl]{A. Hagen}


\ead{ahagen@purdue.edu}


\author[mfarl]{K. Fischer}


\author[salabs]{B. Archambault}


\author[salabs]{S. Rogers}


\author[mfarl,salabs]{R. Taleyarkhan}


\address[mfarl]{Purdue Metastable Fluids and Advanced Research Laboratory, 3601
Sagamore Parkway, Suite H., Lafayette, IN, 47905}


\address[salabs]{Sagamore Adams Laboratory, LLC., 3601 Sagamore Parkway, , Lafayette,
IN, 47905}
\begin{abstract}
\lipsum[1-2]
\end{abstract}
\maketitle

\section{Introduction}

The improvement of neutron detection is central to health physics,
homeland security, and even fission power. The following will discuss
Tension Metastable Fluid Detectors (TMFDs\nomenclature{TMFD}{Tension Metastable Fluid Detectors})
as an alternative to state of the art detectors. It will then discuss
a current optimization and design effort using multiphysics simulation
as the core. Finally, it will discuss the accuracy the simulation
is able to achieve and how that has been used to further the current
system.


\subsection{Motivation}

The National Academy of Engineering has long posed prevention of nuclear
terror as a top priority \citet{Engineering2012}. To prevent nuclear
terror, one must monitor the transport and use of special nuclear
material (SNM\nomenclature{SNM}{Special Nuclear Material}), which
is defined by the Atomic Energy Act of 1954 \citet{UnitedStatesCongress1954}.
This material gives off telltale neutron signatures, which makes neutron
sensors an ideal way to track SNM. With this in mind, the United States
has installed gamma and neutron detectors in many border portals,
achieving a scanning rate of above 99\% \citet{Gowadia2012}. It is
clear that there is a great demand for neutron detectors in this capacity.

Another field which heavily relies on the use of neutron detectors
is that of health physics. \hphantom{write stuff about health physics here, use a couple sources}

Thus far, currently technology has been able to provide sensors to
fulfill the requirements of both the Department of Homeland Security
in terror prevention, and health physics in dose monitoring. The state
of the art detectors that are used in the aforementioned applications
are those of BF3 detectors and He3 detectors. He-3 or BF3 gas in placed
under a high electrical bias in an evacuated chamber, and upon interaction
in the chamber, ions are created, which are then measured as electrical
pulses, indicating detection events \citet{Knoll2000}. Unfortunately,
these detectors have many practical detriments for the applications
proposed. First, these chambers also detect gamma events, and may
even be saturated by these events. Second, they are extremely expensive,
especially with no He3 being currently produced. Third, the detection
systems are extremely complicated and require several expensive electrical
components. Finally, in the case of He3 detectors, the NRC has stated
that they will not consider any new neutron detection technology using
He3 \hphantom{quote nrc}.

The TMFD is a system that can alleviate most, if not all, of the shortcomings
of state of the art neutron detectors. In short, the TMFD uses tensile
pressures to convert neutron events within a working fluid into macroscopic,
visible bubbles. The tensile pressure brings the fluid so close to
the spinodal limit that a neutron event will cause violent cavitation.
The visibility and audibility of these detection events removes the
need for all but a modest electrical drive train. The small probability
for interactions of gammas in the working fluid makes the TMFD gamma
insensitive up to $10^{11}\,\frac{\gamma}{cm^{2}\cdot s}$ experimentally
(equivalent to $1\, m$ from a spent fuel rack), and $10^{22}\,\frac{\gamma}{cm^{2}\cdot s}$
theoretically (equivalent to the center of a research reactor). The
sensitive fluids are generally organic solvents and are widely available
on the market. Along with the readily available fluids, the entire
system can be made for \$100-\$1000 of material cost, a vast decrease
from state of the art detectors. A comparison of the TMFD system to
state of the art systems is provided in the oft-published Table \ref{tab:Comparion between TMFD System and He-3 and BF-3 Systems}.

\begin{table}


\caption{Comparion between TMFD System and He-3 and BF-3 Systems\label{tab:Comparion between TMFD System and He-3 and BF-3 Systems}}


\centering{}%
\begin{tabular}{>{\centering}m{2.1cm}>{\centering}m{2.5cm}>{\centering}m{2.5cm}}
\hline 
Parameter & Conventional System & TMFD System\tabularnewline
\hline 
Intrinsic Efficiency & $~0\%$ ($MeV$ neutrons) 

$~90\%$ ($0.01\, eV$ neutrons)

{\tiny{($3\, cm$ x $30\, cm$ tube)}} & $~90\%$ ($MeV$ neutrons)

$~90\%$ ($0.01\, eV$ neutrons)

{\tiny{($10\, cm$ x $10\, cm$ tube)}}\tabularnewline
On-Off times & Minutes; saturation during pulsed interrogation & Microseconds; adaptable for pulsed systems\tabularnewline
Gamma Blindness? & Limited; saturation in high gamma fields & Completely\tabularnewline
Neutron Directionality? & No with single systems; Yes if arrays are used & Yes (to within $10^{o}$) with single system\tabularnewline
Cost & High ($\sim\$5k$-$\$10k$ for single tube systems) & Low-to-Modest ($\$50$-$\$1k+$)\tabularnewline
\hline 
\end{tabular}
\end{table}


The TMFD system requires the evoluation and sustainment of a tensile
pressure field. This can be done in several different ways, two of
which are the centrifugal method (the technology on which the Centrifugally
Tensioned Metastable Fluid Detector - CTMFD\nomenclature{CTMFD}{Centrifugally Tensioned Metastable Fluid Detector}
- is based) and the acoustic method (which drives the Acoustically
Tensioned Metastable Fluid Detector - ATMFD\nomenclature{ATMFD}{Acoustically Tensioned Metastable Fluid Detector}).
By using a diamond shaped glass piece with a bulb full of fluid at
the bottom, the CTMFD is able to utilize centrifugal force into stretching
the fluid from the center outwards. This provides a stable, quadratically
varying pressure field from the center of rotation outwards. The ATMFD
requires the ultrasonic vibration of the fluid cavity, which under
resonant conditions, provides a temporally sinusoidal, spatially complex
pressure field throughout the resonant cavity.

Because of the method of tensioning, the ATMFD outperforms the CTMFD
for the purposes of neutron detection. This is mainly because of three
factors. First, the CTMFD is rendered completely insensitive following
a detection event. The system must be shut down and brought back up
to speed before being sensitive again ($\sim5-30\, s$). The ATMFD
is oscilating at anywhere from $20\, kHz$ to $80\, kHz$, and thus
is reset and sensitive again within microseconds. Second, the CTMFD
is generally small in size, and thus has a smaller geometric probablity
of interaction than the ATMFD. For comparison, the CTMFD has between
$3\, cm^{3}$ and $23\, cm^{3}$, whereas a general approximation
of the sensitive volume of the CTMFD is a $100\, cm^{3}$ volume.
Lastly, the large sensitive volume compared to the event size in the
ATMFD allows for ``multiplicity'' events, where two neutrons are
detected in tandem. This can be an indicator of certain nuclear signatures
and may be a way to ensure simulants have not been used to counteract
security measures. The CTMFD is unable to do any multiplicity. It
is for these important features that the ATMFD is an important technology
to develop.

In order to develop the ATMFD technology, optimization of the resonant
chamber parameters to gain the highest sensitive volumes is needed.
The parameter space of the chamber in enormous, with at least 15 geometric
variables in the current iteration of the chamber, all varying in
frequency space, and all sensitve to the working fluid and ambient
temperature. Because of this enormous parameter space, a multiphysics
simulation has been developed and validated for use in sensitive volume
maximization, faster prototyping, and design leadership for future
iterations.


\subsection{Background}

Although well known within the nuclear engineering literature, the
concept of metastable fluid neutron and alpha detection may not be
fully understood by those in the applied acoustics field. A cursory
introduction to the detection mechanism will be provided. Following
this introduction, a description of the current acoustically tensioned
detection system will be provided, including the oscillator used.
Finally, discussion on the simulation of this system will be introduced,
providing the method and boundary values used to provide profilimetry.
The discussion of profilimetry will also give insight into how a Monte
Carlo based nuclear particle transport and energy deposition code
could be further used to estimate detector efficiency.

As stated earlier, the introduction of small amounts of energy to
a tensed fluid can cause a macroscopic cavitation event. This energy
may be imparted by neutron collisions, alpha collisions, or other
similar energy deposition events. The initial energy deposition occurs
on a femtometer scale, and through the collision mechanism is propagated
to picometer or nanometer scales. If the cavitation reaches this size,
called the critical radius

\section{Methodology}

\lipsum[21]


\subsection{Validation}

\lipsum[22-27]


\subsection{Simulation}

\lipsum[28-33]


\subsection{Optimization}

\lipsum[34-39]


\section{Results}

\lipsum[40]


\subsection{Economical ATMFD Design}

\lipsum[1-5]


\subsection{Economical ATMFD Performance Characteristics and Comparison to Simulation}

\lipsum[6-10]


\subsection{Conical ATMFD Design}

\lipsum[11-16]


\subsection{Conical ATMFD Performance Characteristics and Comparison to Simulation}

\lipsum[17-22]


\section{Discussion}

\lipsum[23-33]

\bibliographystyle{elsarticle-num-names}
\bibliography{X:/INOK_User_Files/ahagen/Papers/Bibliographies/Masters_Thesis,/media/lab/INOK_User_Files/ahagen/Papers/Bibliographies/Masters_Thesis}

\end{document}
